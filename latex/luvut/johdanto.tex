\chapter{Johdanto} \label{Johdanto}

Sovelluksen tehokas muistinhallinta on keskeisessä osassa sulavan käyttökokemuksen takaamisessa. Sulautettujen järjestelmien tuomat haasteet korostavat muistinhallinnan merkistystä entisestään. Muistin, prosessorin ja laitteiston komponenttien ominaisuuksien rajallisuus asettavat kehittäjälle haasteita, joiden ratkaisut voivat vaatia kehittäjältä hyvinkin kustomoituja ja vaativia rakenteita, jos vertaillaan PC-tietokoneille kehitettävien ohjelmien muistin rakennetta.

Kirjallisuuskatsauksessa tullaan tutkimaan sulautettujen järjestelmien muistinahallintaa näiden rajoitteiden vaikuttaessa. Päätutkimuskysymyksenä on "Millaisia muistinhallinnan tekniikoita voidaan hyödyntää sulautetuissa järjestelmissä". Katsauksessa tullaan käsittelemään sovelluksen muistin ja muistinhallinnan teoriaa, ja perustietoa sulautetuista järjestelmistä. Näiden käsitteiden ymmärtäminen on keskeistä varsinaisten muistinhallinnan tekniikoiden ja rakenteiden ymmärtämisessä. Kirjallisuuskatsaus keskittyy kehittäjän omiin henkilökohtaisiin ratkaisuihin ohjelmointikieli työkalunaan. Katsauksessa ei tulla käymään läpi ohjelmointikielien sisäisiä muistin allokointiominaisuuksia ja -algoritmeja. Tämä rajaus sivuuttaa muutamia tärkeitä aihepiirejä tämän katsauksen ympärillä, kuten mm. roskien keruun.

Tietoa on haettu mm. Google Scholarista, IEEE Xplore:sta sekä ACM Digital Librarysta, ja sitä on haettu hakulauksekkeella: “embedded system” AND (”memory allocat*” OR “memory manag*”) AND (technique* OR method* OR solution*).
