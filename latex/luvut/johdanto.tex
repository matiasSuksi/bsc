\chapter{Johdanto} \label{Johdanto}

Sovelluksen tehokas muistinhallinta on keskeisessä osassa sulavan käyttökokemuksen takaamisessa. Sulautettujen järjestelmien tuomat haasteet korostavat muistinhallinnan merkitystä entisestään. Muistin, prosessorin ja laitteiston komponenttien ominaisuuksien rajallisuus asettavat kehittäjälle haasteita, joiden ratkaisut voivat vaatia kehittäjältä hyvinkin kustomoituja ja vaativia ohjelmakoodin rakenteita, jos vertaillaan perinteisille henkilökohtaisille tietokoneille kehitettävien ohjelmien muistin rakennetta.

Kirjallisuuskatsauksessa tullaan tutkimaan sulautettujen järjestelmien muistinhallintaa näiden rajoitteiden vaikuttaessa. Päätutkimuskysymyksenä on "millaisia muistinhallinnan tekniikoita voidaan hyödyntää sulautetuissa järjestelmissä". Katsauksessa tullaan käsittelemään sovelluksen muistin ja muistinhallinnan teoriaa, ja perustietoa sulautetuista järjestelmistä. Näiden käsitteiden ymmärtäminen on keskeistä varsinaisten muistinhallinnan tekniikoiden ja rakenteiden ymmärtämisessä.

Kirjallisuuskatsaus keskittyy kehittäjän omiin henkilökohtaisiin ratkaisuihin ohjelmointikieli työkalunaan. Tietokoneiden resurssien virtualisointi on yleistynyt nykypäivänä hajautettujen järjestelmien ja pilvipalveluiden tullessa yhä yleisimmiksi, mutta tässä kirjallisuuskatsauksessa rajaamme aihepiirin käsittämään perinteiseen RAM-muistin hallintaan liittyviä konsepteja. Virtualisoidun muistin allokoiminen tullaan sivuuttamaan kokonaan. Lisäksi, katsauksessa ei tulla käymään läpi kuin ainoastaan pintapuoleisesti ohjelmointikielien ja kääntäjien sisäisiä muistin allokointiominaisuuksia ja -algoritmeja. Tämä rajaus sivuuttaa muutamia merkittäviä aihepiirejä, kuten mm. roskien keruun ohjelmointikielissä. Katsauksessa on valittu esimerkkiohjelmointikieleksi C konseptien havainnollistamiseksi. Valinta on perusteltua, sillä C on yksi yleisimmistä ohjelmointikielistä sulautetuissa järjestelmissä. Suurimmassa osassa katsauksessa käsiteltävissä artikkeleissa myös implisiittisesti oletetaan, että ohjelmointikieli, jonka ympärille muistirakenteita toteutetaan on juuri C.

Taustoitusta varten tietoa on haettu Google Scholarista, IEEE Xplore:sta sekä ACM Digital Librarysta, ja sitä on haettu hakulauksekkeella: “embedded system” AND (”memory allocat*” OR “memory manag*”) AND (technique* OR method* OR solution*). Varsinaisia muistinhallinan tekniikoita varten tietoa on haettu myös suoraan konseptien omilla nimillä.

Kirjallisuuskatsauksessa kaksi lukua ovat taustoittavia teoria lukuja, jotka ovat tärkeitä varsinaisen tutkimuskysymyksen takana olevien käsitteiden ymmärtämiseen. Toinen luku on teoriaa sovelluksen muistin, ja sen hallinnasta ja kolmas luku käsittelee sulautettuja järjestelmiä. Neljännessä luvussa pureudutaan tarkemmin varsinaisiin muistinhallinnan tekniikoihin ja esitellään yksityiskohtaisemmin muistirakenteita ja miten niitä voidaan hyödyntää sulautetuissa järjestelmissä. Viidennessä luvussa tehdään havaintoja näistä muistirakenteista ja mitä näiden toteutuksessa pitää ottaa huomioon sulautetuissa järjestelmissä. Yhteenvedossa kootaan yhteen työn havainot ja pyritään vastaamaan asetettuun tutkimuskysymykseen.