\chapter{Johdanto} \label{Johdanto}

Sovelluksen tehokas muistinhallinta on keskeisessä osassa sulavan käyttökokemuksen takaamisessa. Sulautettujen järjestelmien tuomat haasteet korostavat muistinhallinnan merkistystä entisestään. Muistin, prosessorin ja laitteiston komponenttien ominaisuuksien rajallisuus asettavat kehittäjälle haasteita, joiden ratkaisut voivat vaatia kehittäjältä hyvinkin kustomoituja ja vaativia rakenteita, jos vertaillaan perinteisille PC-tietokoneille kehitettävien ohjelmien muistin rakennetta.

Kirjallisuuskatsauksessa tullaan tutkimaan sulautettujen järjestelmien muistinahallintaa näiden rajoitteiden vaikuttaessa. Päätutkimuskysymyksenä on "millaisia muistinhallinnan tekniikoita voidaan hyödyntää sulautetuissa järjestelmissä". Katsauksessa tullaan käsittelemään sovelluksen muistin ja muistinhallinnan teoriaa, ja perustietoa sulautetuista järjestelmistä. Näiden käsitteiden ymmärtäminen on keskeistä varsinaisten muistinhallinnan tekniikoiden ja rakenteiden ymmärtämisessä. Kirjallisuuskatsaus keskittyy kehittäjän omiin henkilökohtaisiin ratkaisuihin ohjelmointikieli työkalunaan. Tietokoneiden resurssien virtualisointi on yleistynyt nykypäivänä hajautettujen järjestelmien ja pilvipalveluiden tullessa yhä yleisimmiksi, mutta tässä kirjallisuuskatsauksessa rajaamme aihepiirin käsittämään perinteiseen RAM-muistin hallintaan liittyviä konsepteja. Virtualisoidun muistin allokoiminen tullaan sivuuttamaan kokonaan. Lisäksi, katsauksessa ei tulla käymään läpi kuin ainoastaan pintapuoleisesti ohjelmointikielien ja kääntäjien sisäisiä muistin allokointiominaisuuksia ja -algoritmeja. Tämä rajaus sivuuttaa muutamia merkittäviä aihepiirejä, kuten mm. roskien keruun ohjelmointikielissä. Katsauksessa on valittu esimerkkiohjelmointikieleksi C konseptien havainnollistamiseksi. Valinta on perusteltua C:n alkuperäisen luonteen vuoksi sekä se on yksi yleisimmistä ohjelmointikielistä sulautetuissa järjestelmissä. Suurimmassa osassa katsauksessa käsiteltävissä artikkeleissa myös implisiittisesti oletetaan, että ohjelmointikieli, jonka ympärille muistirakenteita toteutetaan on juuri C. Lisäksi monet esimerkit C:llä voidaan yleistää C:n laajennukselle C++:lle, joka on myös yksi yleisimmistä kielistä sulautetuissa järjestelmissä.

Taustoitusta varten tietoa on haettu Google Scholarista, IEEE Xplore:sta sekä ACM Digital Librarysta, ja sitä on haettu hakulauksekkeella: “embedded system” AND (”memory allocat*” OR “memory manag*”) AND (technique* OR method* OR solution*). Varsinaisia muistinhallinan tekniikoita varten tietoa on haettu myös suoraan konseptien omilla nimillä.