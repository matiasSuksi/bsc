\chapter{Sulautetut järjestelmät} \label{Kolmas luku}

\section{Mikä on sulautettu järjestelmä}

Sulautetulle järjestelmälle on hyvin vaikeaa antaa yksikäsitteistä määritelmää, mutta yleisesti sulautetuilla järjestelmillä viitataan näkymättömiin ja ubiikkeihin tietokoneisiin, jotka ovat suunniteltu jonkin spesifisen toiminnallisuuden suorittamiseen. Esimerkkejä sulautetuista järjestelmistä ovat mm. autot. Autoissa polttoaineen syöttöä, automaattista jarrutusjärjestelmää ja navigointijärjestelmää ohjaa loppujen lopuksi tietokone.\cite{rtcfes2015book} Nämä esimerkit symboloivat hyvin sulautetun järjestelmän yleistä kuvausta. Perinteistä henkilökohtaisia tietokonetta käyttäessään, käyttäjä on käytännössä jatkuvasti tietoinen siitä, että hän käyttää tietokonetta, mutta autoa ajaessaan kuljettaja ei tätä jatkuvasti tiedosta eikä hänen tarvitse sitä tiedostaa.

\subsection{Reaaliaikanen sulautettu järjestelmä}

Reealiaikainen sulautettu järjestelmä (engl. \textit{real-time embedded system}) on yleinen käsite johon törmää usein sulautetuista järjestelmistä puhuttaessa. Reaaliaikainen sulautettu järjestelmä on sulautettu järjestelmä, joka vastaa järjestelmän ulkopuolisiin tapahtumiin reaaliajassa. Tämä tarkoittaa, että sulautettu järjestelmä kykenee havaitsemaan ulkoisen tapahtuman, pystyy reagoimaan ja prosessoimaan tapahtuman sekä tuottamaan tarvittavan tuloksen tietyssä aikarajassa.\cite{rtcfes2015book} Reaaliaikaisuuden käsite on hyvin tärkeä tässä tutkimuksessa, sillä lähdekartoituksessa huomattiin, että suurin osa tutkielman aihepiiriin liittyvissä artikkeleissa muistinhallinnan tekniikoita pyritään toteuttamaan nimenomaan juuri reaaliaikaiseen sulautettuun järjestelmään.

\section{Sulautettujen järjestelmien haasteet} 

Sulautettujen järjestelmien periaate tietokoneesta, joka on luotu nimenomaisesti jonkin spesifin toiminnallisuuden suorittamiseen tehokkaasti, aiheuttaa rajoitteita järjestelmän toteutuksen suunnitteluun. Keskeisiä rajoitteita ovat mm. tuotantokustannukset, virrankulutus ja tietokoneen fyysinen koko. Lisäksi itse sulautettujen järjestelmien kehittäjällä on oltava monipuolista osaamista monilta eri teknologioiden osa-alueilta.\cite{rtcfes2015book} Aikaisemmin mainitut rajoitteet aiheuttavat sen, että sulautetuissa järjestelmissä monesti muistin määrä on hyvin rajallinen, ja muistinhallinnan mahdollistavien komponenttien ominaisuudet ovat hyvin rajallisia. Lisäksi, sulautetulle järjestelmälle asetetut vaatimukset, kuten aikaisemmin mainittu reaaliaikaisuus, jo valmiiksi rajoitetuilla resursseilla, tekevät kehitystyöstä entistä haastavampaa. Nämä haasteet ovat keskeinen osa kirjallisuuskatsauksen analyysiä, ja muistinhallinnan tekniikoita peilataan juuri näiden rajoitteiden aiheuttamiin haasteisiin.

Perinteisten henkilökohtaisien tietokoneiden prosessorien rakenne on monimutkainen, ja ne tarjoavat monipuolisesti ominaisuuksia monipuolisten tehtävien suorittamisen. Sulautetuissa järjestelmissä prosessorit ovat usein huomattavasti yksinkertaisempia, sillä ne ovat optimoitu juuri tietyn tehtävän suorittamista varten. Modernit prosessorit sisältävät hyvin usein sisäänrakennetun muistinhallintayksikön (engl. \textit{memory management unit, MMU}), joka suojaa muistia ja tarjoaa virtuaalimuistin moniajoa varten. Sulautetun järjestelmän prosessorissa muistinhallintayksikköä ei välttämättä ole ollenkaan.\cite{rtcfes2015book} Tämä on esimerkki rajoitteesta, jolloin ei ole riittävää pohtia ainoastaan menetelmiä miten muistinkäyttöä voitaisiin tehostaa järjestelmän nopeuden ja muistin rajallisuuden puolesta, vaan nyt kehittäjä joutuu pohtimaan ratkaisua komponenttien rajallisten ominaisuuksien näkökulmasta.