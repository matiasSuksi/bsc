\chapter{Sulautetut järjestelmät} \label{Kolmas luku}

\section{Mikä on sulautettu järjestelmä}

Sulautetulle järjestelmälle on hyvin vaikeaa antaa yksikäsitteistä määritelmää, mutta yleisesti sulautetuilla järjestelmällä viitataan näkymättömiin ja ubiikkeihin tietokoneisiin, jotka ovat suunniteltu jonkin spesifisen toiminnallisuuden suorittamiseen. Esimerkkejä sulautetuista järjestelmistä ovat mm. autot. Autoissa polttoaineen syöttöä, automaattista jarrutusjärjestelmää ja navigointijärjestelmää ohjaa loppujen lopuksi tietokone.\cite{rtcfes2015book} Nämä esimerkit symboloivat hyvin sulautetun järjestelmän yleistä kuvausta. Perinteistä PC-tietokonetta käyttäessään, käyttäjä on käytännössä jatkuvasti tietoinen siitä, että hän käyttää tietokonetta, mutta autoa ajaessaan kuljettaja ei tätä jatkuvasti tiedosta eikä hänen tarvitsekkaan sitä tiedostaa.

\section{Sulautettujen järjestelmien haasteet} 

Sulautettujen järjestelmien periaate tietokoneesta, joka on luotu nimenomaisesti jonkin spesifin toiminnallisuuden suorittamiseen tehokkaasti, aiheuttaa rajoitteita järjestelmän toteutuksen suunnitteluun. Keskeisiä rajoitteita ovat mm. tuotantokustannukset, virrankulutus ja tietokoneen fyysinen koko. Lisäksi itse sulautettujen järjestelmien kehittäjän on omattava monipuolisesti osaamista monilta eri teknologioiden osa-alueilta\cite{rtcfes2015book} Edm. rajoitteet aiheuttavat sen, että sulautetuissa järjestelmissä monesti muistin määrä on hyvin rajallinen ja sen hallinnan mahdollistavien komponenttien ominaisuudet ovat hyvin rajallisia. Tämä on keskeinen osa kirjallisuuskatsauksen analyysiä, ja muistinhallinan tekniikoita tullaan peilaamaan juuri tämän rajoitteen aiheuttamiin haasteisiin.

Perinteisten PC-tietokoneiden prosessorien rakenne on monimutkainen, ja ne tarjoavat monipuolisesti ominaisuuksia monipuolisten tehtävien suorittamisen. Sulautetuissa järjestelmissä prosessorit ovat usein huomattavasti yksinkertaisempia, sillä ne ovat optimoitu juuri tietyn tehtävän suorittamista varten. Modernit prosessorit sisältävät hyvin usein sisäänrakennetun muistinhallintayksikön (engl. \textit{memory management unit, MMU}), joka suojaa muistia ja tarjoaa virtuaalimuistin moniajoa varten. Sulautetun järjestelmän prosessorissa muistinhallintayksikköä ei välttämättä ole ollenkaan.\cite{rtcfes2015book} Tämä on esimerkki rajoitteesta, jolloin ei ole riittävää pohtia ainoastaan menetelmiä miten muistinkäyttöä voitaisiin tehostaa järjestelmän nopeuden ja muistin rajallisuuden puolesta, vaan nyt kehittäjä joutuu pohtimaan ratkaisua komponenttien rajallisten ominaisuuksien pohjalta.