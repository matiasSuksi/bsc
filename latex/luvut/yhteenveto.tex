\chapter{Yhteenveto} \label{Yhteenveto}

Tutkielmassa esitellään sovelluksen muistin teoriaa, joka on johdattelua itse sovelluksen muistinhallintaan. Tutkielmassa myös esitellään ja määritellän sulautetun järjestelmän käsite, ja tuodaan esille mitä haastetia sulautetut järjestelmät aiheuttavat kehittäjälle. Tutkielma päättyy erilaisten muistinhallintatekniikoiden esittelyyn, niiden analysointiin ja suorituskyvyn mittaamiseen esittelyyn. Tutkielman edetessä huomataan, että staattinen muistinhallinta aiheuttaa tietynlaisia rajoitteita, jonka takia monesti jäädään keskittymään dynaamisien menetelmien optimointiin.

Tutkielma ei kykene antamaan yhtä oikeaa vastausta päätutkimuskysysmykseen: "millaisia muistinhallinnan tekniikoita voidaan hyödyntää sulautetuissa järjestelmissä". Sulautettuja järjestelmiä on paljon erilaisia ja oikeanlainen tehokas muistinkäytön ratkaisu on täysin tapauskohtainen. Ratkaisua suunnitellessa on otettava huomioon järjestelmän laitteisto, vaatimukset ja käyttökohde. Vaikka tutkielma ei kykene antamaan yksikäsitteistä vastausta ongelmaan, tutkielma esittelee faktoja mitä pitää ottaa huomioon muistinhallintatekniikoiden soveltamisessa sulautetuissa järjestelmissä ja tuo esiin mahdollisia eteentulevia haasteita. Toiseen tutkimuskysymykseen saavutetaan tutkielmassa huomattavasti parempi vastaus, ja tutkielmassa havaitaan, että se onkin vahvasti sidoksissa päätutkimuskysymykseen. Tutkielmassa havaitut päähaasteet ovat juuri edm. mainitut laitteiston rajoitteet, järjestelmän vaatimukset ja käyttökohde. Tutkielmassa mainittuja laitteiston rajoitteita ovat mm. tuotantokustannukset, virrankulutus ja laitteiston fyysinen koko. On tärkeää muistaa, että rajoitteita on muitakin ja monet rajoitteet syntyvät johdannaisina näistä kolmesta keskeisestä rajoitteesta tai niiden yhdistelmästä. Tästä esimerkkinä mainittakoon, että mm. tietokoneen fyysinen koko ja tuotantokustannukset aiheuttavat näistä johtuvan rajoitteen kuin järjestelmän rajallinen muistin määrä.

Tukielman aiheeseen liittyvää käsitteistöä on hyvin paljon, minkä takia tämä tutkielma on hyvin teoriapainoitteinen pintapuolinen katsaus aiheeseen. Tutkielmaa voisi laajentaa tulevaisuudessa mm. kokeellisella tutkimuksella, jossa testattaisiin eri muistinhallintatekniikoiden suorituskykyä erityppisissä sulautetuissa järjestelmissä ja toteuttamalla yksityiskohtaisempaa analyysiä muistinhallintatekniikoista. Kattavammalla tutkielmalla voitaisiin saavuttaa jonkinlainen yleistys, että minkälaiset tekniikat ovat tehokkaita tietyntyyppisissä sulautetussa järjestelmässä.