\chapter{Yhteenveto} \label{Yhteenveto}

Tutkielma esitteli sovelluksen muistin teoriaa, joka on johdattelua itse muistinhallintaan. Seuraavaksi tutkielmassa esitellään ja määritellän sulautetun järjestelmän käsite, ja tuodaan esille mitä haastetia sulautetut järjestelmät aiheuttavat kehittäjälle. Tutkielma päättyy erilaisten muistinhallintatekniikoiden esittelyyn, niiden analysointiin ja suorituskyvyn mittaamiseen esittelyyn. Tutkielman edetessä huomataan, että staattinen muistinhallinta aiheuttaa tietynlaisia rajoitteita, jonka takia monesti jäädään keskittymään dynaamisien menetelmien optimointiin. 

Tutkielmassa ei kyetty antamaan yhtä oikeaa vastausta päätutkimuskysysmykseen: "millaisia muistinhallinnan tekniikoita voidaan hyödyntää sulautetuissa järjestelmissä". Sulautettuja järjestelmiä on paljon erilaisia ja oikeanlainen tehokas muistinkäytön ratkaisu on täysin tapauskohtainen. Ratkaisua suunnitellessa on otettava huomioon järjestelmän laitteisto, vaatimukset ja käyttökohde. Vaikka tutkielma ei kyennyt antamaan yksikäsitteistä vastausta ongelmaan, tutkielma esittelee mahdollisia seikkoja mitä pitää ottaa huomioon muistinhallintatekniikoiden soveltamisessa sulautetuissa järjestelmissä, ja tuo esiin mahdollisia eteentulevia haasteita.

Tukielman aiheeseen liittyvää käsitteistöä on hyvin paljon, minkä takia tämä tutkielma on hyvin teoriapainoitteinen pintapuolinen katsaus aiheeseen. Tutkielmaa voisi laajentaa mm. kokeellisella tutkielmalla, jossa testattaisiin eri muistinhallintatekniikoiden suorituskykyä erityppisissä sulautetuissa järjestelmissä, ja toteuttamalla yksityiskohtaisempaa analyysiä muistinhallinantekniikoista.