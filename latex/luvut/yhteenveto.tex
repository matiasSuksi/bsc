\chapter{Yhteenveto} \label{Yhteenveto}


1. Sulautettujen järjestelmien luonteesta aiheutuvien haasteisiin vastaaminen?
    Kehittäjä joutuu pohtimaan toteutusta näiden asioiden pohjalta, ja niiden johdannaisista?
        Virrankulutus
        Fyysinenkoko --> Teho ja muistin määrä rajallinen
        Tuotantokustannukset --> Kompomentit yksinkertaisia --> vähemmän ominaisuuksia
    Onko järjestelmä reaaliaikainen? Tämä oli hyvin usein artikkeleissa, että oletuksena on reaaliaikainen systeemi.

    Riippuu täysin järjestelmän käyttötarkoituksesta --> Onko kyseessä auto, mikro, lentokone, ilmanvaihtokone --> Mitä nämä vaativat toteutukselta

    Näitä kehittäjä joutuu pohtimaan ja kussakin tilanteessa tehokkain mahdollinen toteutus on täysin näistä seikoista kiinni eikä voida tehdä minkäänlaista yleistystä, mikä menetelmä/algoritmi on sulautetuissa järjestelmissä tehokkain.


~\\

2. Millaisia algoritmeja?
    Monet artikkelit lähtivät lähtökohdasta, että halutaan toteuttaa staattinen menetelmä, sillä ne ovat nopeampia. Kuitenkin monesti joudutaan tyytymään dynaamiseen ratkaisuun, sillä staattisen allokoinnin luonne aiheuttaa liian paljon rajoituksia tehokkaan ratkaisun tuottamiseen. Täten käsitellyt artikkelit ja niin myös tämä tutkielma, keskittyy dynaamisen menetelmiin, niiden kehittämiseen ja optimointiin.

    Staattisen allokoinnin suurin etu on se, että sillä kyetään vastaamaan reaaliaikaisuuden asettamiin haasteisiin parhaiten ja staattiset menetelmät ovat stabiilimpia kuin dynaamiset. Staattiset allokointimenetelmät ennaltaehkäisevät muistin fragmentoitumista.\cite{daroemmfera@2006}

    Uusia tapoja tehostaa muistinkäyttöä, kääntäjän merkitys muistinkäytössä?

    Artikkelien esittelemät algoritmit pyrkivät vähentämään fragmentoitumista dynaamisissa menetelmissä, yksi merkittävimmistä haasteista suoraan algoritmien pohjalta.
