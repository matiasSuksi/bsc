\chapter{Muistinhallinan tekniikoita ja rakenteita} \label{Neljäs luku}

Tässä luvussa tullaan esittelemään yleisiä muistinhallinan tekniikoita, joita voidaan hyödyntää sulautetuissa järjestelmissä.

\section{Rengaspuskuri}

Rengaspuskuri (engl. \textit{circular buffer} on järjestetty tietorakenne, jossa viimeisen alkion jälkeen palataan takaisin ensimmäiseen alkioon. Yleensä rengaspuskuri toteutetaan, joko järjestettynä taulukkona tai linkitettynä listana, jonka viimeinen alkio osoittaa takaisin ensimmäiseen alkioon. Rengaspuskurin etuindeksi (engl. \text{front index}) osoittaa tyhjän paikan, johon seuraavaksi lisättävä alkio laitetaan. Takaindeksi (engl. \text{back index}) osoittaa seuraavaksi poistettavan alkion paikan. Rengaspuskurit ovat erittäin yleisiä tietorakenteita juuri reealiaikaisissa sulautetuissa järjestelmissä, joissa useat prosessit kommunikoivat keskenään. Rengaspuskuri toimii väliaikaisena muistina prosesseille, jolloin prosessit voivat toimia asynkronisesti.\cite{c2015book}

Seuraavaksi esitellään yksinkertaisen kokonaislukuja sisältävän rengaspuskurin totetus.

\begin{algorithm}[tbh]
\begin{lstlisting}[language=C]
typedef struct RengasPuskuri_t {
    int* taulukko;  //Osoitin taulukkoon
    int koko;       //Maksimikoko
    int alkioiden_lukumaara;    //
    int ensimmainen_alkio;  //Indeksi puskurin alkuun
    int viimeinen_alkio;    //Indeksi puskurin loppuun
}
\end{lstlisting}
\caption{Rengaspuskurin implementaatio\label{alg:Rengaspuskuri}}
\end{algorithm}

\section{Buddy Memory Allocation}
\section{Memory Pooling}
\section{Fixed-size Block Allocation}
\section{Memory Banks}
\section{Object Pools}
\section{Slab allocation}