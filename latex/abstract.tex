\keywords{sulautettu järjestelmä, muistin allokointi, staattinen allokointi, dynaaminen allokointi, keko, pino, reaaliaikaisuus}

\begin{abstract}

Sovelluksen tehokas muistinkäyttö on tärkeää sujuvan käyttökokemuksen takaamiseksi, ja sulautetuissa järjestelmissä tämä sujuvuus korostuu entisestään. Tutkielmassa tutkitaan minkälaisia muistinhallinnan tekniikoita voidaan hyödyntää sulautetuissa järjestelmissä. Tutkielmassa käydään läpi sovelluksen muistin rakennetta, esitellään mitä rajoitteita ja haasteita sulautetut järjestelmät aiheuttavat tehokkaalle muistinhallinnalle.

Tutkielmassa havaitaan, että staattiset muistinhallinnan menetelmät ovat monesti suorituskyvyn kannalta tehokkaampia kuin dynaamiset menetelmät. Kuitenkin monesti sulautettujen järjestelmien vaatimukset aiheuttavat sen, kuten reaaliaikaisuus, että joudutaan tyytymään dynaamiseen muistinhallintaan. Tutkielma ei kykene tuottamaan yleistystä, mitkä muistinhallintatekniikka ovat tehokkaimpia sulautetuissa järjestelmissä, vaan tutkielmassa havaitaan, että muistihallintatekniikan sovellus on täysin riippuvainen sovelluskohteesta. Täten myös tutkielmassa esitellyt muistinhallinnan tekniikat ovat valittu tutkielmaan lähdekartoituksessa useimmiten vastaan tulleina muistirakenteita, joita sulautettuun järjestelmään on pyritty soveltamaan. Tutkielmassa havaittuja sulautettujen järjestelmien haasteita ovat mm. laitteiston rajoitteet, järjestelmän vaatimukset ja käyttökohde. Muitakin muistinhallintatekniikoita ja -rakenteita on olemassa paljon, ja näitä onkin syytä tutkia mahdollisissa jatkotutkimuksissa.

\end{abstract}

