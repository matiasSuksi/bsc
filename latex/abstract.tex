\keywords{sulautettu järjestelmä, muistin allokointi, staattinen allokointi, dynaaminen allokointi, keko, pino, reaaliaikaisuus}

\begin{abstract}

Sovelluksen tehokas muistinkäyttö on tärkeää sujuvan käyttökokemuksen takaamiseksi, ja sulautetuissa järjestelmissä tämä sujuvuus korostuu entisestään. Tämä tutkielma pyrkii etsimään vastauksen tutkimuskysymykseen: "millaisia muistinhallinnan tekniikoita voidaan hyödyntää sulautetuissa järjestelmissä". Tutkielmassa käydään läpi sovelluksen muistin rakennetta, esitellään mitä rajoitteita ja haasteita sulautetut järjestelmät aiheuttavat tehokkaalle muistinhallinnalle.

Tutkielma ei kykene tuottamaan yleistystä, mitkä muistinhallintatekniikka ovat tehokkaimpia sulautetuissa järjestelmissä, vaan tutkielmassa havaitaan, että muistihallintatekniikan sovellus on täysin riippuvainen sovelluskohteesta. Täten myös tutkielmassa esitellyt muistinhallinan tekniikat ovat valittu tutkielmaan lähdekartoituksessa useimmiten vastaan tulleina muistirakenteita, joita sulautettuun järjestelmään on pyritty kussakin artikkelissa soveltamaan. Muitakin muistinhallintatekniikoita ja -rakenteita löytyi paljon ja näitä onkin syytä tutkia mahdollisissa jatkotutkimuksissa.

\end{abstract}

